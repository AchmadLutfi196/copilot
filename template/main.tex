\documentclass[10pt,a4paper]{article}

% Required packages
\usepackage[utf8]{inputenc}
\usepackage[T1]{fontenc}
\usepackage{mathptmx} % Times New Roman font
\usepackage[bahasa]{babel} % Indonesian language support
\usepackage{geometry}
\usepackage{setspace}
\usepackage{titlesec}
\usepackage{tocloft}
\usepackage{fancyhdr}
\usepackage{graphicx}
\usepackage{amsmath}
\usepackage{amsfonts}
\usepackage{amssymb}
\usepackage{cite}
\usepackage{url}
\usepackage{hyperref}
\usepackage{booktabs}
\usepackage{array}
\usepackage{longtable}

% Page setup
\geometry{
    a4paper,
    left=3cm,
    right=3cm,
    top=3cm,
    bottom=3cm
}

% Line spacing
\onehalfspacing

% Header and footer
\pagestyle{fancy}
\fancyhf{}
\fancyhead[L]{\fontsize{9}{11}\selectfont Kajian Pustaka Sistem Pendukung Keputusan}
\fancyhead[R]{\fontsize{9}{11}\selectfont \thepage}
\renewcommand{\headrulewidth}{0.5pt}

% Title formatting
\titleformat{\section}
{\normalfont\large\bfseries}{\thesection}{1em}{}
\titleformat{\subsection}
{\normalfont\normalsize\bfseries}{\thesubsection}{1em}{}
\titleformat{\subsubsection}
{\normalfont\normalsize\bfseries}{\thesubsubsection}{1em}{}

% Table of contents formatting
\renewcommand{\cftsecfont}{\bfseries}
\renewcommand{\cftsecpagefont}{\bfseries}

% Hyperref setup
\hypersetup{
    colorlinks=true,
    linkcolor=black,
    filecolor=magenta,      
    urlcolor=blue,
    citecolor=black,
    pdftitle={Kajian Pustaka Sistem Pendukung Keputusan untuk Bantuan Pendidikan},
    pdfauthor={Nama Mahasiswa},
    pdfsubject={Literature Review},
    pdfkeywords={Sistem Pendukung Keputusan, Bantuan Pendidikan, SAW, Neural Network}
}

% Document information - EDIT THIS SECTION
\title{
    \vspace{-2cm}
    \Huge\textbf{KAJIAN PUSTAKA}\\
    \vspace{0.5cm}
    \Large\textbf{SISTEM PENDUKUNG KEPUTUSAN}\\
    \Large\textbf{UNTUK BANTUAN PENDIDIKAN}\\
    \vspace{1cm}
    \normalsize\textit{Literature Review on Decision Support Systems for Educational Aid}
}

\author{
    \textbf{[NAMA LENGKAP MAHASISWA]}\\
    \textbf{[NIM]}\\
    \vspace{0.5cm}
    \textbf{Program Studi [NAMA PROGRAM STUDI]}\\
    \textbf{Fakultas [NAMA FAKULTAS]}\\
    \textbf{Universitas [NAMA UNIVERSITAS]}\\
    \vspace{1cm}
    \today
}

\date{}

\begin{document}

% Title page
\maketitle
\thispagestyle{empty}
\newpage

% Table of contents
\tableofcontents
\thispagestyle{empty}
\newpage

% Start page numbering
\setcounter{page}{1}

% Introduction section
\section{PENDAHULUAN}

% TODO: Tuliskan pendahuluan kajian pustaka Anda di sini
% Jelaskan:
% - Latar belakang pentingnya sistem pendukung keputusan untuk bantuan pendidikan
% - Tujuan kajian pustaka
% - Ruang lingkup kajian
% - Struktur penulisan

% Paper 1 Section
\section{KAJIAN PAPER 1: METODE SIMPLE ADDITIVE WEIGHTING}

\subsection{Metode yang Digunakan}

% TODO: Jelaskan metode Simple Additive Weighting (SAW) yang digunakan
% Sertakan:
% - Definisi dan konsep dasar SAW
% - Langkah-langkah algoritma SAW
% - Formula dan perhitungan yang digunakan
% - Proses implementasi dalam konteks bantuan pendidikan

\subsection{Dataset yang Digunakan}

% TODO: Deskripsikan dataset yang digunakan dalam penelitian
% Sertakan informasi tentang:
% - Jumlah dan karakteristik sampel data
% - Kriteria-kriteria yang digunakan untuk penilaian
% - Bobot masing-masing kriteria
% - Sumber data dan metode pengumpulan data
% - Preprocessing data yang dilakukan

\subsection{Hasil Eksperimen dan Analisis}

% TODO: Ringkas hasil eksperimen dan analisis
% Sertakan:
% - Hasil implementasi sistem
% - Tingkat akurasi atau efektivitas sistem
% - Perbandingan dengan metode manual (jika ada)
% - Analisis performa sistem
% - Tabel atau grafik hasil (jika diperlukan)

\subsection{Kelebihan dan Kekurangan}

\subsubsection{Kelebihan}
\begin{itemize}
    \item % TODO: Tuliskan kelebihan-kelebihan metode SAW
    \item % TODO: Kemudahan implementasi
    \item % TODO: Transparansi proses
    \item % TODO: Dan lain-lain...
\end{itemize}

\subsubsection{Kekurangan}
\begin{itemize}
    \item % TODO: Tuliskan kekurangan-kekurangan yang ditemukan
    \item % TODO: Keterbatasan metode
    \item % TODO: Masalah yang perlu diatasi
    \item % TODO: Dan lain-lain...
\end{itemize}

\subsection{Referensi yang Digunakan}

% TODO: Analisis referensi-referensi utama yang digunakan dalam paper
% Jelaskan relevansi dan kualitas referensi terhadap topik penelitian

% Paper 2 Section
\section{KAJIAN PAPER 2: METODE NEURAL NETWORK}

\subsection{Metode yang Digunakan}

% TODO: Jelaskan metode Neural Network yang digunakan
% Sertakan:
% - Arsitektur neural network yang digunakan
% - Algoritma pembelajaran (training algorithm)
% - Fungsi aktivasi yang dipilih
% - Proses training dan testing
% - Parameter-parameter yang digunakan

\subsection{Dataset yang Digunakan}

% TODO: Deskripsikan dataset untuk neural network
% Sertakan:
% - Karakteristik dan ukuran dataset
% - Preprocessing data yang dilakukan
% - Pembagian data training dan testing
% - Fitur-fitur input yang digunakan
% - Normalisasi atau transformasi data

\subsection{Hasil Eksperimen dan Analisis}

% TODO: Ringkas hasil eksperimen neural network
% Sertakan:
% - Hasil training neural network
% - Tingkat akurasi prediksi
% - Analisis confusion matrix (jika ada)
% - Perbandingan performa dengan metode lain
% - Grafik learning curve atau performa

\subsection{Kelebihan dan Kekurangan}

\subsubsection{Kelebihan}
\begin{itemize}
    \item % TODO: Kelebihan metode Neural Network
    \item % TODO: Kemampuan menangani pola kompleks
    \item % TODO: Akurasi prediksi yang tinggi
    \item % TODO: Dan lain-lain...
\end{itemize}

\subsubsection{Kekurangan}
\begin{itemize}
    \item % TODO: Kekurangan yang ditemukan
    \item % TODO: Kompleksitas implementasi
    \item % TODO: Kebutuhan data training yang besar
    \item % TODO: Dan lain-lain...
\end{itemize}

\subsection{Referensi yang Digunakan}

% TODO: Analisis referensi-referensi utama dalam paper neural network
% Jelaskan kualitas dan relevansi referensi

% Comparative Analysis
\section{ANALISIS KOMPARATIF}

\subsection{Perbandingan Metodologi}

% TODO: Bandingkan kedua metode
% Buat tabel perbandingan yang mencakup:
% - Kompleksitas implementasi
% - Kebutuhan data dan preprocessing
% - Transparansi proses pengambilan keputusan
% - Akurasi dan reliabilitas hasil
% - Waktu komputasi
% - Kemudahan interpretasi

\subsection{Kesesuaian Konteks Aplikasi}

% TODO: Analisis kesesuaian masing-masing metode
% Pertimbangkan faktor:
% - Kemudahan penggunaan bagi stakeholder
% - Kebutuhan interpretabilitas hasil
% - Skalabilitas sistem
% - Maintenance dan update sistem
% - Biaya implementasi

% Conclusion
\section{KESIMPULAN}

% TODO: Tuliskan kesimpulan kajian pustaka
% Sertakan:
% - Ringkasan temuan utama dari kedua penelitian
% - Perbandingan efektivitas kedua metode
% - Rekomendasi penggunaan metode berdasarkan konteks
% - Kontribusi untuk bidang ilmu
% - Saran untuk penelitian lanjutan

% Bibliography
\bibliographystyle{plain}
\bibliography{references}

\end{document}